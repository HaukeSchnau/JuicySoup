\documentclass[12pt]{report}
\usepackage[ngerman]{babel}

\begin{document}
\begin{titlepage}
    \centering
    {\scshape\LARGE IGS Lilienthal \par}
	\vspace{1cm}
	{\scshape\Large Informatik \par}
	\vspace{1.5cm}
	{\huge\bfseries JuicySoup Dokumentation \par}
	\vspace{2cm}
	{\Large\itshape Leon Kommerau, Pascal Jung, \\Hauke Schnau, Amro El-Seyed \par}
	\vfill
	Lehrer:\par
	Herr \textsc{Engelbertz}

	\vfill

    % Bottom of the page
	{\large \today\par}
    \end{titlepage}

    \tableofcontents


    \chapter{Unser Spiel}

    \section{Die Idee}

    \section{Anleitung}

    \section{Warum serious?}


    \chapter{Grundlagen der Algorithmik}

    \section{Was ist ein Algorithmus?}

    \section{Was ist ein Programm?}

    \section{Was braucht man zum Programmieren?}
    Zum Programmieren braucht man einen Computer, auf dem zumindest ein
    Textbearbeitungsprogramm und ein Compiler oder Interpreter installiert
    sind. Programmiert man wie wir in JavaScript für den Browser, reicht
    dafür ein normaler Webbrowser aus. Für Java oder C++ bräuchte man allerdings
    einen Compiler, der den Code so übersetzt, dass er für den Computer
    verständlich ist.

    \section{Was ist p5.js?}
    p5.js ist eine Bibliothek, die auf JavaScript aufbaut und eigene Funktionen
    zum Zeichnen von geometrischen Formen und Bildern auf ein Canvas implementiert.
    Sie ist besonders an Künstler, Designer, Lehrer und Anfänger gerichtet.

    \section{Was sind Variablen und Datentypen?}

    \section{Was sind Kontrollstrukturen?}
    Kontrollstrukturen beeinflussen den Ablauf eines Programms.

    Dazu gehören zum Beispiel Entscheidungen wie \texttt{if/else}-Abfragen, die einen Codeblock nur dann
    ausführen, wenn eine gegebene Bedingung zutrifft. Der Code im optionalen \texttt{else}-Block
    wird dann ausgeführt, wenn die Bedingung \textbf{nicht} zutrifft.

    Schleifen führen einen Codeblock mehrfach aus. Es gibt sie in Form von
    \texttt{while}- und \texttt{for}-Schleifen. Beide führen den Code so lange aus,
    bis die im Kopf der Schleife gegebene Bedingung \textbf{nicht} mehr \texttt{true} ist,
    wobei die \texttt{for}-Schleife dies meist durch eine Zählervariable erreicht.
    Dazu gibt es noch die \texttt{for-each} Schleife, die über ein Array iteriert.

    \section{Was sind Funktionen und Objekte?}
    Funktionen

    \chapter{Organisation der Gruppenarbeit}

    \section{Verlauf der Projektarbeit}

    \section{Probleme}

    \section{Aufgabenverteilung}


    \chapter{Fazit}

    \chapter{Quellenverzeichnis}
    
\end{document}