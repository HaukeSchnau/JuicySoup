\documentclass[12pt]{report}
\usepackage[ngerman]{babel}

\usepackage{hyperref}
\hypersetup{
    colorlinks=true,
    linkcolor=blue,
    filecolor=magenta,      
    urlcolor=cyan,
}

\begin{document}
\begin{titlepage}
    \centering
    {\scshape\LARGE IGS Lilienthal \par}
	\vspace{1cm}
	{\scshape\Large Informatik \par}
	\vspace{1.5cm}
	{\huge\bfseries JuicySoup Dokumentation \par}
	\vspace{2cm}
	{\Large\itshape Leon Kommerau, Pascal Jung, \\Hauke Schnau, Amro El-Seyed \par}
	\vfill
	Lehrer:\par
	Herr \textsc{Engelbertz}

	\vfill

    % Bottom of the page
	{\large \today\par}
    \end{titlepage}

    \tableofcontents


    \chapter{Unser Spiel}

    \section{Die Idee}
    Juicy Soup ist ein Spiel mit historischem Inhalt. Man spielt einen Zeitreisenden,
    der in der Vergangenheit gestrandet ist und versucht, wieder in die Gegenwart
    zu kommen. Dafür muss er Gegenstände aus der jeweiligen Zeit finden, um
    seine Zeitmaschine zu reparieren.

    \section{Anleitung}
    Nachdem man ein Level ausgewählt hat, kann man seinen Charakter mit den
    Tasten A und D nach links und rechts steuern. Mit der Leertaste springt man,
    mit Shift duckt man sich. Wenn man eine Waffe ausgewählt hat, kann man mit der
    Maus in Richtung des Mauszeigers schießen.
    Das Ziel ist auf der rechten Seite.

    \section{Warum serious?}
    Das Spiel ist serious, weil man beim Spielen auf spielerische Art
    etwas über die dargestellten Zeitalter lernt. Wir möchten den Spielern
    mit spaßigen Methoden mehr über die damaligen Helden und Bösewichte näher
    bringen.

    \chapter{Grundlagen der Algorithmik}

    \section{Was ist ein Algorithmus?}
    Ein Algorithmus ist eine Folge von Anweisungen, die eine vorher bestimmte
    Aufgabe erledigen. Er muss bei denselben Eingabewerten immer die selben 
    Ausgabewerte produzieren.

    \section{Was ist ein Programm?}
    Ein Programm ist eine Ansammlung von Algorithmen, die zusammenwirken.
    Der Nutzer/Anwender des Programms kann kann ihm beispielsweise über
    Kommandozeilenargumente oder eine Schnittstelle wie Maus oder Tastatur
    Eingabewerte übermitteln.

    \section{Was braucht man zum Programmieren?}
    Zum Programmieren braucht man einen Computer, auf dem zumindest ein
    Textbearbeitungsprogramm und ein Compiler oder Interpreter installiert
    sind. Programmiert man wie wir in JavaScript für den Browser, reicht
    dafür ein normaler Webbrowser aus. Für Java oder C++ bräuchte man allerdings
    einen Compiler, der den Code so übersetzt, dass er für den Computer
    verständlich ist.

    \section{Was ist p5.js?}
    p5.js ist eine Bibliothek, die auf JavaScript aufbaut und eigene Funktionen
    zum Zeichnen von geometrischen Formen und Bildern auf ein Canvas implementiert.
    Sie ist besonders an Künstler, Designer, Lehrer und Anfänger gerichtet.

    \section{Was sind Variablen und Datentypen?}
    In Variablen kann ein Programm Werte speichern, verändern und abrufen.
    Die Variablen haben dabei immer einen Datentyp wie z.B. \texttt{String},
    \texttt{boolean}, \texttt{int}, \texttt{Array} und viele mehr. Bei den
    meisten Programmiersprachen muss man den Datentyp beim deklarieren der
    Variable mit angeben. Das ist bei JavaScript jedoch nicht nötig.

    \section{Was sind Kontrollstrukturen?}
    Kontrollstrukturen beeinflussen den Ablauf eines Programms.

    Dazu gehören zum Beispiel Entscheidungen wie \texttt{if/else}-Abfragen, die einen Codeblock nur dann
    ausführen, wenn eine gegebene Bedingung zutrifft. Der Code im optionalen \texttt{else}-Block
    wird dann ausgeführt, wenn die Bedingung \textbf{nicht} zutrifft.

    Schleifen führen einen Codeblock mehrfach aus. Es gibt sie in Form von
    \texttt{while}- und \texttt{for}-Schleifen. Beide führen den Code so lange aus,
    bis die im Kopf der Schleife gegebene Bedingung \textbf{nicht} mehr \texttt{true} ist,
    wobei die \texttt{for}-Schleife dies meist durch eine Zählervariable erreicht.
    Dazu gibt es noch die \texttt{for-each} Schleife, die über ein Array iteriert.

    \section{Was sind Funktionen und Objekte?}
    Funktionen werden genutzt, um Code zu strukturieren und wiederzuverwenden.
    Ein Ablauf, der oft gleich oder sehr ähnlich ausgeführt wird, kann in eine
    Funktion abgekapselt werden, um den Code übersichtlicher und leichter
    lesbar zu machen.

    Objekte sind ähnlich wie Funktionen nützlich, um Struktur in den Code zu 
    bringen. Sie werden genutzt, um ähnliche Variablen in einer zu bündeln und
    bestehen aus Key/Value paaren. Ein Objekt, das ein Rechteck beschreibt,
    kann die Attribute \texttt{x}, \texttt{y}, \texttt{breite}, 
    \texttt{höhe} und \texttt{farbe} haben.

    \chapter{Organisation der Gruppenarbeit}

    \section{Verlauf der Projektarbeit}
    Nachdem wir gemeinsam unsere Idee festgelegt hatten, schrieben wir die
    ersten Zeilen. Der Prototyp, der dabei entstand ähnelte einem Super Mario
    Spiel. Nachdem erste Kollisionsabfragen und Physik implementiert waren,
    war es Zeit für einen Map-Editor, damit man die Level nicht von Hand in eine
    JSON-Datei schreiben muss. Dann wurden Grafiken für das Spiel rausgsucht,
    Level designt und Items, Gegner und Helfer einprogrammiert. Und natürlich
    wurden währenddessen ständig auch noch Bugs gefixt.

    \section{Aufgabenverteilung}
    \begin{itemize}
        \item Hauke
        \begin{itemize}
            \item Großteil des Codes geschrieben
            \item Kapitel 2 ,,Grundlagen der Algorithmik'' verfasst
        \end{itemize}
        \item Pascal
        \begin{itemize}
            \item Grafiken rasugesucht
            \item Level Design
            \item Fazit verfasst
        \end{itemize}
        \item Leon
        \begin{itemize}
            \item Musik rausgesucht
            \item Teile des Codes geschrieben
            \item Kaptitel 1 ,,Unser Spiel'' verfasst
        \end{itemize}
        \item Amro
        \begin{itemize}
            \item Namensschöpfer
            \item Kapitel 3 ,,Organisation der Gruppenarbeit'' verfasst
            \item mentale Unterstützung
        \end{itemize}
    \end{itemize}

    \chapter{Fazit}
    Wir sind mit unserem Ergebnis sehr zufrieden, auch wenn wir nicht alle
    Level fertigstellen konnten. Leider muss man zurzeit noch springen, um
    den Speer werfen zu können.
    
    Unser größtes Problem war, dass wir uns so viel
    vorgenommen hatten, dass wir nicht alles in dem Zeitraum erledigen konnten.
    Ein weiteres Hindernis waren die Grafiken, da keiner von uns künstlerisch
    sehr begabt ist. Der Mangel an Grafiken lag nicht zuletzt daran, dass unsere
    Arbeitsteilung zumindest zu Beginn miserabel war. Sobald wir uns jedoch
    jeden Tag online getroffen haben, ging es voran und alle arbeiteten am
    Projekt. Wir nutzten das Live Share Plugin von Visual Studio Code, damit wir
    alle gleichzeitig Code schreiben, das Spiel spielen und Assets einfügen konnten.
    Die größte technische Hürde war die Kollisionserkennung. Sowohl die zwischen zwei
    achsenparallelen Rechtecken, besonders aber die zwischen rotierten Rechtecken
    machte uns zu Schaffen.

    Wenn wir an dem Spiel weiterarbeiten könnten, würden wir alle weiteren
    Gegner, NPCs und Level in das Spiel einfügen.

    \chapter{Quellenverzeichnis}
    \url{https://p5js.org}
    \url{https://www.codeproject.com/Articles/15573/2D-Polygon-Collision-Detection}
    
\end{document}